\documentclass[twoside,11pt]{article}

\usepackage{blindtext}

% Any additional packages needed should be included after jmlr2e.
% Note that jmlr2e.sty includes epsfig, amssymb, natbib and graphicx,
% and defines many common macros, such as 'proof' and 'example'.
%
% It also sets the bibliographystyle to plainnat; for more information on
% natbib citation styles, see the natbib documentation, a copy of which
% is archived at http://www.jmlr.org/format/natbib.pdf

% Available options for package jmlr2e are:
%
%   - abbrvbib : use abbrvnat for the bibliography style
%   - nohyperref : do not load the hyperref package
%   - preprint : remove JMLR specific information from the template,
%         useful for example for posting to preprint servers.
%
% Example of using the package with custom options:
%
% \usepackage[abbrvbib, preprint]{jmlr2e}

\usepackage{jmlr2e}

% Definitions of handy macros can go here

\newcommand{\dataset}{{\cal D}}
\newcommand{\fracpartial}[2]{\frac{\partial #1}{\partial  #2}}

% Heading arguments are {volume}{year}{pages}{date submitted}{date published}{paper id}{author-full-names}

\usepackage{lastpage}
\jmlrheading{23}{2025}{1-\pageref{LastPage}}{1/21; Revised 5/22}{9/22}{21-0000}{Yaoshiang Ho}

% Short headings should be running head and authors last names

\ShortHeadings{Supervised Learning Preference Optimization}{Yaoshiang Ho}
\firstpageno{1}

\begin{document}

\title{Supervised Learning Preference Optimization: Rethinking RLHF and DPO as Supervised Learning}

\author{\name Yaoshiang Ho \email yaoshiang@gmail.com \\
      %  \addr Department of Statistics\\
      %  University of Washington\\
      %  Seattle, WA 98195-4322, USA
      %  \AND
      %  \name Author Two \email two@cs.berkeley.edu \\
      %  \addr Division of Computer Science\\
      %  University of California\\
      %  Berkeley, CA 94720-1776, USA
      }

\editor{}

\maketitle 

\begin{abstract}%   <- trailing '%' for backward compatibility of .sty file
Direct Policy Optimization (DPO) is a popular approach to aligning 
large language models with human preferences. 
In this paper, we analyze the underlying math and 
propose a new algorithm
which we call Supervised Learning Preference Optimization (SLPO). 
 
\end{abstract}

\begin{keywords}
  Reinforcement Learning from Human Feedback (RLHF), Direct Policy Optimization (DPO)
\end{keywords}

\section{Introduction}

Alignment is the task of ensuring that the behavior of a
Large Language Model (LLM) is consistent 
with human preferences. 

A key difference between the alignment phase and
other phases of training an LLM is that the alignment phase considers
full sequences of text, rather than simply predicting the next token, as
in the pretraining and supervised fine-tuning (SFT) phases. 

The alignment approach popularized by the commercial success of ChatGPT was 
Reinforcement Learning from Human Feedback (RLHF, \cite{ouyang2022training}). 
Despite its effectiveness, RLHF requires training a second model, called
a reward model, as well as Proximal Policy Optimization (PPO), resulting in a 
technique that is more complex than the basic supervised learning. It also
requires a Kullback-Leibler (KL) divergence term to regularize the
changes to the LLM during alignment training.

Direct Policy Optimization (DPO) is a simpler approach to alignment
which does not require a secondary reward model. In the paper introducing
DPO, the authors examine the underlying
approach of RLHF and propose
the DPO objective to align the target LLM directly using
maximum likelihood estimation (MLE). 
The key insight from the DPO paper is that an LLM's
outputs can be reparameterized into a reward model using ratios, logs,
and the Bradley-Terry model \cite{bradley1952rank}

The specific contribution of this paper is to reframe the alignment
phase away from reward modeling entirely and treating it simply as
a pure supervised learning problem by training a model to align to
a directly modified probability distribution. We call this
approach Supervised Learning Preference Optimization (SLPO).

\section{Related Work}

Related work\dots

\section{Preliminaries}

We review and continue the analysis of the DPO objective by its 
author.

The DPO objective is defined as follows:

\begin{equation}
  L_\mathrm{DPO}(\pi_\theta; \pi_\mathrm{ref}) =
  \underbrace{
  -\mathbb{E}_{(x, y_w, y_l) \sim D} 
  \log }_{1} 
  \left[ 
    \underbrace{\sigma }_{2}
    \left(
    \underbrace{\beta \log \frac{\pi_\theta(y_w \mid x)}{\pi_\mathrm{ref}(y_w \mid x)}}_{3}
    - \underbrace{\beta \log \frac{\pi_\theta(y_l \mid x)}{\pi_\mathrm{ref}(y_l \mid x)}}_{4} 
    \right)
  \right].
  \end{equation}  

In the underbraced section 1, we see the standard 
negative log likelihood (NLL) objective.
In the underbraced section 2, we see the Bradley-Terry model 
\footnote{A ranking method that is mathematically equivalent and perhaps more 
widely understood is the ELO score, used to rank Chess players, 
and, LLMs in the Chatbot Arena \cite{elo1978rating,chiang2024chatbot}. 
Both ELO and Bradley-Terry assign scores to players, and 
pass the difference through a sigmoid function to assess the probability of 
the LHS player of winning. }.
In the underbraced sections 3 and 4, we see how the 
reference and language model's predictions are reparameterized
into a winning and losing score: they are the 
log of the ratio of the language model to the reference model, for the winning and 
losing completion, respectively. 
This score is later described as the reward function.

With simple algebraic manipulation, we can rewrite this objective as:

\begin{equation}
  \label{eq:reg}
  L_\mathrm{DPO}(\pi_\theta; \pi_\mathrm{ref}) =
  \underbrace{
  -\mathbb{E}_{(x, y_w, y_l) \sim D} 
  \log }_{1} 
  \left[ 
    \underbrace{\sigma }_{2}
    \left(
    \underbrace{\beta \log \frac{\pi_\theta(y_w \mid x)}{\pi_\theta(y_l \mid x)}}_{3}
    - \underbrace{\beta \log \frac{\pi_\mathrm{ref}(y_w \mid x)}{\pi_\mathrm{ref}(y_l \mid x)}}_{4} 
    \right)
  \right].
  \end{equation}  

We can interpret the undercomponents as follows: The first underbrace is the NLL, as before. 
The second underbrace is the
Bradley-Terry model, as before. 
The third underbrace is the log of the ratio
language models probability of the winner divided by the loser. This ratio 
is optimized to be bigger, given that the log, sigmoid, and log from underbraces
1, 2, and 3 are all monotonic functions. Since a ratio of probabilities is
an odds ratio, we will refer to this as the 
language model's log-odds ratio. 

The fourth underbrace is reference model's log-odds ratio. This
is a constant per $y_w$ and $y_l$ and is not differentiated. However,
these values and their ratio vary across different $y_w$ and $y_l$ - that is,
each row of training data will have a different value for this constant. 
More specifically, since $\pi_\theta$ is initialized as $\pi_\mathrm{ref}$, 
the difference between underbraces 3 and 4 starts at zero during training, 
and the shape of the sigmoid function (underbrace 2) and its gradient are
known. During training, the language model's log-odds ratio will increase,
however, the sigmoid function will naturally regularize and decelerate
the increase. This is the exact outcome described by the DPO authors 
in their analysis of the gradient of DPO. \emph{But we have developed a different
intuition the DPO loss: rather than reparameterizing the language model's
output into a reward model, we are simply regularizing the optimization of 
the language model's log-odds ratio.}

Let's analyze the regularization effect. The DPO authors investigated
the gradient of the sequence of tokens... let's go two steps further
by considering each token individually, and the logit behind it. 

Our variable $y$ is a sequence of tokens. More formally, it is

\begin{equation}
  \label{eq:joint}
  \pi_\mathrm{ref}(y \mid x) = \prod_{t=1}^T \pi_\mathrm{ref}(y_t \mid x, y_{<t}),
\end{equation}
where \(y = (y_1, y_2, \ldots, y_T)\) is the output sequence, 
\(x\) is the input context, and \(y_{<t} = (y_1, y_2, \ldots, y_{t-1})\) 
represents the tokens generated prior to time step \(t\). The term 
\(\pi_\mathrm{ref}(y_t \mid x, y_{<t})\) denotes the conditional 
probability of generating token \(y_t\) given the input \(x\) and 
the previously generated tokens \(y_{<t}\).

Since both $\pi$ language model are LLMs, they are activated
using the softmax function. Logits can be normalized using the
log-softmax function, allowing them to be exponentiated
to yield the probability of the token, avoiding the need for
a softmax function and its reliance on the entire probability 
distribution.

Let us establish the term $\mathrm{g}$ for the layers of the model up to the softmax
activation, namely, the feature extractor and log-softmax normalization: 

\begin{equation}
  \pi(y \mid x) = \exp (\mathrm{g}(y \mid x))
\end{equation}

where

\begin{equation}
  \label{eq:g}
  \mathrm{g}(y \mid x) = \mathrm{logsoftmax}(\mathrm{f}(y \mid x))
\end{equation}

Plugging Equations ~\ref{eq:joint} and ~\ref{eq:g} into the DPO objective, 
with a focus just on the score
inside the sigmoid of the Bradley-Terry model, we have:

\[
    \beta \log \frac
    {\prod_{t=1}^{T_w} \exp (\mathrm{g_\theta}(y_{w,t} \mid x, y_w{<t}))}
    {\prod_{t=1}^{T_w} \exp (\mathrm{g_\mathrm{ref}}(y_{w,t} \mid x, y_w{<t}))}
    - 
    \beta \log \frac
    {\prod_{t=1}^{T_l} \exp (\mathrm{g_\theta}(y_{l,t} \mid x, y_w{<t}))}
    {\prod_{t=1}^{T_l} \exp (\mathrm{g_\mathrm{ref}}(y_{l,t} \mid x, y_w{<t}))}
    \\
\]

\[
\beta \left( 
\sum_{t=1}^{T_w} \left[ \mathrm{g_\theta}(y_{w,t} \mid x, y_{w,<t}) - \mathrm{g_\mathrm{ref}}(y_{w,t} \mid x, y_{w,<t}) \right] 
- \sum_{t=1}^{T_l} \left[ \mathrm{g_\theta}(y_{l,t} \mid x, y_{l,<t}) - \mathrm{g_\mathrm{ref}}(y_{l,t} \mid x, y_{l,<t}) \right] 
\right)
\]

This formulation of the DPO objective provides additional insight. 
Each logit of the language model, which was originally conceived to predict the probability of a token,
has been normalized by the logit of the reference model via a shift (this normalization
should not to be confused with the normalization
by the log-softmax function, which was purely for mathematical convenience). The 
sum of these normalized logits is then passed through a standard sigmoid activation and 
negative log likelihood function.

The starting value for the difference between the language model 
and the reference model's logits is zero, since the language model is initialized
by the reference model, so the gradient passed to each logit on the first step of 
optimization is -0.5 for the winning token and 0.5 for the losing token. As the 
winning sequence increases its odds, and the losing sequence decreases its odds,
the the gradient will decrease, as is well known for the negative log loss applied
to a sigmoid activated function (e.g. classic binary crossentropy loss for
logistic regression). Although sigmoid is not an additive function, 
e.g. $\sigma(a + b) \neq \sigma(a) + \sigma(b)$, essentially as the logit of
each winning token increases and each losing token decreases, the magnitude of the
gradient applied to all logits decreases.

And yet... the DPO equation has a lot of machinery for what ends up being
a standard negative log likelihood loss, applied to a normalized logit.
Is there a way to further simplify
the normalization, eliminating the need for the Bradley-Terry model, the 
log of ratios, and the concept of reward functions? 

\section{Supervised Learning Preference Optimization}

Section body

Here is a citation \cite{chow:68}.

\section{Results}

Section body

Here is a citation \cite{chow:68}.

\section{Conclusion}

Section body

Here is a citation \cite{chow:68}.

% Acknowledgements and Disclosure of Funding should go at the end, before appendices and references

\acks{The author thanks Chiara Cerini, Peter Tran, and Sam Wookey for their 
invaluable reviews of early drafts of this work.}

% Manual newpage inserted to improve layout of sample file - not
% needed in general before appendices/bibliography.

\newpage

\appendix
\section{}
\label{app:theorem}

% Note: in this sample, the section number is hard-coded in. Following
% proper LaTeX conventions, it should properly be coded as a reference:

%In this appendix we prove the following theorem from
%Section~\ref{sec:textree-generalization}:

[appendix]

\section{}

[appendix]
\noindent
{\bf Proof}. We use the notation:

{\noindent \em Remainder omitted in this sample. See http://www.jmlr.org/papers/ for full paper.}


\vskip 0.2in
\bibliography{sample}

\end{document}
